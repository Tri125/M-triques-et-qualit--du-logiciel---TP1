% !TEX TS-program = pdflatex
% !TEX encoding = UTF-8 Unicode

% This is a simple template for a LaTeX document using the "article" class.
% See "book", "report", "letter" for other types of document.

\documentclass[11pt, french]{article} % use larger type; default would be 10pt
\usepackage[francais]{babel}
\usepackage[utf8]{inputenc} % set input encoding (not needed with XeLaTeX)
\usepackage[T1]{fontenc}
\usepackage[squaren,Gray]{SIunits}
\usepackage{amsmath}
\usepackage{hyperref}
%%% Examples of Article customizations
% These packages are optional, depending whether you want the features they provide.
% See the LaTeX Companion or other references for full information.

%%% PAGE DIMENSIONS
\usepackage{geometry} % to change the page dimensions
\geometry{a4paper} % or letterpaper (US) or a5paper or....
% \geometry{margin=2in} % for example, change the margins to 2 inches all round
% \geometry{landscape} % set up the page for landscape
%   read geometry.pdf for detailed page layout information

\usepackage{graphicx} % support the \includegraphics command and options

% \usepackage[parfill]{parskip} % Activate to begin paragraphs with an empty line rather than an indent

%%% PACKAGES
\usepackage{booktabs} % for much better looking tables
\usepackage{array} % for better arrays (eg matrices) in maths
\usepackage{paralist} % very flexible & customisable lists (eg. enumerate/itemize, etc.)
\usepackage{verbatim} % adds environment for commenting out blocks of text & for better verbatim
\usepackage{subfig} % make it possible to include more than one captioned figure/table in a single float
% These packages are all incorporated in the memoir class to one degree or another...

%%% HEADERS & FOOTERS
\usepackage{fancyhdr} % This should be set AFTER setting up the page geometry
\pagestyle{fancy} % options: empty , plain , fancy
\renewcommand{\headrulewidth}{0pt} % customise the layout...
\lhead{}\chead{}\rhead{}
\lfoot{}\cfoot{\thepage}\rfoot{}

%%% SECTION TITLE APPEARANCE
\usepackage{sectsty}
\allsectionsfont{\sffamily\mdseries\upshape} % (See the fntguide.pdf for font help)
% (This matches ConTeXt defaults)

%%% ToC (table of contents) APPEARANCE
\usepackage[nottoc,notlof,notlot]{tocbibind} % Put the bibliography in the ToC
\usepackage[titles,subfigure]{tocloft} % Alter the style of the Table of Contents
\renewcommand{\cftsecfont}{\rmfamily\mdseries\upshape}
\renewcommand{\cftsecpagefont}{\rmfamily\mdseries\upshape} % No bold!

%%% END Article customizations

%%% The "real" document content comes below...

\title{TP1-IFT3913 HIVER 2014\\- Rapport -}
\author{Tristan Savaria 1062837 (savt09129202) IBrahima Faye 959315 (fayi29129204)}
\date{9 février 2014} 

\begin{document}
\maketitle
\newpage
\tableofcontents
\newpage
\section{Introduction}
\paragraph{}
Dans ce travail pratique, nous avions pour but d'extraire un ensemble de mesures de code Java à partir d'un Workspace.
Pour cela des squelettes de programmes nous ont été fournis pour nous facilité la tâche.
Pour notre part nous avions aussi implémenté une nouvelle classe nommée Metrique sur laquelle on peut
obtenir toutes les mesures extraites en un seul string sous la forme exigée pour ce projet. En outre nous avons effectué 
plusieurs affichages sur la console pour la tracabilité. La quasi-totalité des mesures réussies y ont été affichées.
Cependant nous n'avions pas réussis à obtenir l'intégral des métriques demandées dans la feuille d'instruction.

\section{Métrique Implémentées}
Classe : Nom qualifié de la classe (incluant le nom de package)
\newline
LOC : Nombre de lignes
\newline
NbreInterfaces : Nombre Interfaces de la classes (toutes, transitivement héritées)
\newline
NbreAttrImpl : Nombre d'attributs implantés dans la classe
\newline
PctAttrPub : Pourcentage des attributs qui sont publiques
\newline
PctAttrHer : Pourcentage des attributs hérités
\newline
NbreMethImpl : Nombre de méthodes implantées
\newline
PctMethPub : Pourcentage des méthodes qui sont publiques
\newline
PctMethHer : Pourcentage des méthodes qui sont héritées
\newline
DIT : Profondeur dans l'arbre d'héritage
\newline
NOC : Nombre de fils
\newline
WMC : Somme des complexités des méthodes. Complexité est la complexité cyclomatique.
\newline
fanOut : Nombre de classes sont utilisées par la classe
\newline

\subsection{Métriques Absentes}
LCOM : Manque de cohésion
\newline
CBO : Couplage entre les objets
\newline
fanIn : Nombre de classes qui utilisent la classe
\newline

\section{Détails et Erreurs}
La majorité des métriques extraites ce passent de commentaires. LOC échoue dans certains cas, mais en majorité il ce comporte comme prévue.
Dans les classes imbriquées, le nom de la classe et le LOC sera faux. Le nom s'avère être le nom de la classe supérieur et le LOC, le nombre de lignes de la classe imbriquée. Également, dans certain cas LOC est -1 puisque nous utiliseons  ASTNode.getStartPosition() et ASTNode.getLength() dans le calcul et si aucune information de la source de la position est enregistré, le résultat sera -1. Nous n'avons pas réussis à déterminer la raison derrière ce comportement.
\newline

Le nombre d'interfaces transitivement hérités comprend des interfaces de librairies java, mais ne comprend pas ceux du type Object.
\newline
 Les constantes ne sont pas pris en compte pour les attributs implémantés de la classe, ni pour le pourcentage des attributs publiques.
 \newline
 La profondeur dans l'arbre d'héritage cesse avant d'atteindre la classe Object.
 \newline
 Pour le WMC, la complexité cyclomatique total a été implémanté. C'est à dire, le WMC d'une classe est équivalent à la somme du WMC de chaque méthodes de la classe.
 \newline
 fanOut a plusieurs erreurs. Il y a une mauvaise filtration de données. Pour le calcul, nous observons le type des attributs de la classe, des variables déclarées dans les méthodes et le type des variables passées en paramêtre dans les méthodes. Pour les attributs d'une classe et ceux déclarés dans le méthodes, les types primitifs sont exclus. Cependant, il aurait été mieux d'exclure le type String et les collection java. Pour les paramètres des méthodes, seulement int, String et char sont exclus. Si un tableau de ces types est utilisé, il sera calculé. Cette mauvaise filtration nous donne une mesure de fanOut innapproprié quoique prévisible si l'on connais l'implémentation.
 \newline

\section{Commentaire}
LCOM, CBO et fanIn sont des mesures nécessitant d'établir une relation entre deux classes. Avec l'Abstract Syntax Tree extrait, il est facile d'établir une relation de parenté ou de voir si notre classe courante (qui est en train d'être visitée) utilise un type étrangé à elle-même. Il n'est pas aussi facile d'établir la relation dans les deux sens puisqu'il faut revisiter l'arbre en entier tout en ayant en mémoire notre visite précédente pour pouvoir en tirer des conclusions.

Ces mesures, ainsi que fanIn et NOC ont été nos difficultés majeures dans ce travail. Une mauvaise connaissance de l'Abstract Syntax Tree et de ses noeuds nous a considérablement ralentit dans notre progression.

\end{document}
